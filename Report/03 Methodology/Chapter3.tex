%!TEX root =  ../Report.tex

\chapter{Experimental Method}                               
\label{sec: Experimental Method}



%==============================================================
%                        section 1. 
%       Problem Identification & Research Questions
\section{Introduction \& Methodology Outline}
\label{sec:meth_framework}
%==============================================================
To address the research question proposed in Section \ref{sec:research_Qs}, this study employs a three-phase experimental framework.

\begin{enumerate}[label={\emph{Phase} \arabic*}]
    \item \textbf{Host-PC Training and Quantisation:}\\ 
    High-fidelity baseline models are developed and trained on a high-performance host environment using the PyTorch framework. This phase will include dataset pre-processing, temporal model training (\ac{lstm}, \ac{gru}, \ac{tcn} and \ac{cnn} hybrids), as well as precision scaling (including \ac{qat}) before exporting teh architectures to an interpretable ONNX format. 

    \item \textbf{Edge Deployment \& Hardware Profiling:}\\
    ONNX models are deployed to an NVIDIA Jetson Orin (4GB). Using NVIDIA TensorRT, the models are executed under strictly constrained hardware power modes while dynamic telemetry (latency, memory footprint, and power consumption) is polled via on-board hardware sensors.

    \item \textbf{Analytical Evaluation Phase:}\\
    Quantitative data generated from the edge device is analysed against the baseline predictive accuracy, establishing a rigorous trade-off profile between SOH predictive fidelity and edge-energy consumption.
    
\end{enumerate}





%==============================================================
%                        section 2. 
%                 Data and Preprocessing
\section{Dataset Acquisition and Preprocessing}
%==============================================================
To ensure the \ac{dl} models are evaluated against realistic, non-linear degradation effects, the benchmark dataset by Lu et al. \cite{lu_battery_2022} was selected. This dataset comprises 77 nominally identical high-energy 18650 lithium-ion batteries. Crucially, it steps beyond standard constant-current constant-voltage (\ac{cccv}) laboratory protocols; 55 of the cells were cycled under arbitrary, uncertain future conditions with randomised charge currents obeying a uniform distribution (1C, 2C, 3C) changing every five cycles. This stochastic variation induces heavy distribution shifts, providing a rigorous test for the sequence models.

\subsection{Feature Extraction: The $Q(v)$ Curve}


%==============================================================
%                        section 3. 
%                 Data and Preprocessing
\section{Model Architectures \& Host-PC Training}
%==============================================================


%==============================================================
%                        section 4. 
%                 Data and Preprocessing
\section{Precision Scaling \& Quantisation}
%==============================================================


%==============================================================
%                        section 5. 
%                     Edge Deployment
\section{Edge Deployment \& Hardware Profiling}
%==============================================================




%==============================================================
%                        section 6. 
%             Evaluation Approach and Metrics 
\section{Evaluation Metrics}
%==============================================================
